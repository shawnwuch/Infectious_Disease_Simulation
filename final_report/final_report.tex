\documentclass{article}

\begin{document}

\section*{Question 4}
\subsection*{Prove, mathematically, that both roots are negative.}

% Enter your answer to Question 4 below this line
Answer: Since a=1 and c=1,
$$b^{2}-4ac = b^{2}-4$$
$$\forall \, 10 \leq b \leq 10^{15}$$
$$10^{2} \leq b^{2} \leq 10^{30}$$ 
so
$$0 < \sqrt{b^{2}-4} < \sqrt{b^{2}}=b$$
therefore,
$$-b+\sqrt{b^{2}-4} < 0$$ and
$$-b-\sqrt{b^{2}-4} < 0$$
hence the two roots
$$x_1=\frac{-b+\sqrt{b^{2}-4ac}}{2a}=\frac{-b+\sqrt{b^{2}-4}}{2} < 0$$
$$x_2=\frac{-b-\sqrt{b^{2}-4ac}}{2a}=\frac{-b-\sqrt{b^{2}-4}}{2} < 0$$


\section*{Question 5}
\subsection*{What is $x_1 x_2$ in terms of $a, b, c$?}

% Enter your answer to Question 5 below this line
Answer: $\frac{c}{a}$.


\section*{Question 6}
\subsection*{Look at \texttt{roots.txt}.  What do you notice about one of the
             roots?}

% Enter your answer to Question 6 below this line
Answer: As b gets larger, one of the roots ($x_1$) approaches 0 and the other one ($x_2$) approaches the value of -b.


\section*{Question 7}
\subsection*{Look at the formula for the quadratic equation for the
             solution $x_1$. For fixed $a$ and $c$, how do the magnitudes of
             terms in the numerator compare as b gets large?}

% Enter your answer to Question 7 below this line
Answer: As b gets large, $\sqrt{b^{2}-4ac} \rightarrow \sqrt{b^{2}} = b$, so the numerator of $x_1$, which is equal to $-b+\sqrt{b^{2}-4ac} \rightarrow -b + b \rightarrow 0$, becomes less negative and approaches 0.


\section*{Question 8}
\subsection*{Given your analysis in 7, discuss what you think is happening in
             the finite precision calculation of $x_1$?}

% Enter your answer to Question 8 below this line
Answer: The difference between b and $\sqrt{b^{2}-4ac}$ becomes too small at large b such that the addition of -b and $\sqrt{b^{2}-4ac}$ gives results that are smaller than machine epsilon ($\epsilon$), thus making $x_1=0$ at large b, even if the true $x_1$ is not 0.

\end{document}
