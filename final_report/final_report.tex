\documentclass{article}
\usepackage{graphicx}
\begin{document}

\title{SDS 392 Intro. to Sci. Programming\\ Final Project: Infectious Disease Simulation}
\author{Chu-Hsiang Wu (cw32746), Petroleum Engineering department, The University of Texas at Austin}
\maketitle

\section*{Contagion}
\subsection*{Run a number of simulations with population sizes and contagion probabilities. Are there cases where people escape getting sick?}

% Enter your answer to Question 4 below this line
Answer:

\subsection*{Incorporate inoculation: read another number representing the percentage of people that has been vaccinated. Choose those members of the population randomly. Describe the effect of vaccinated people on the spread of the disease. Why is this model unrealistic.}

% Enter your answer below this line

\section*{Spreading}
\subsection*{Code the random interactions. Now run a number of simulations varying$\colon$ 1. The percentage of people inoculated, and 2. the chance the disease is transmitted on contact. Record how long the disease runs through the population. With a fixed degree of contagiousness, how is this number of function of the percentage that is vaccinated? Investigate the matter of �herd immunity�: if enough people are vaccinated, then some people who are not will still never get sick. Let�s say you want to have this probability over 95 percent. Investigate the percentage of inoculation that is needed for this as a function of the contagiousness of the disease.}

% Enter your answer below this line
Answer: 



\subsection*{You can make the model more realistic by letting inoculation be only partly effective. For instance, 50\% of people got the flu vaccine, but it was only 40\% effective; 90\% of people have the measles vaccine, and it is about 97\%effective. How does your model function in this case? Keep in mind that different diseases have different degrees of infectiousness.}

% Enter your answer below this line
Answer:


\end{document}
